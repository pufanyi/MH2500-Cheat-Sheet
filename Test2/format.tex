\usepackage{amssymb}
\usepackage{amsthm}
\usepackage{amsmath}
\usepackage{fontspec,xunicode,xltxtra}
\usepackage{titlesec}
\usepackage{indentfirst}
\usepackage{xeCJK}
\usepackage{fancyhdr}
\usepackage{graphicx}
\usepackage{listings}
\usepackage{printlen}
\usepackage{ifthen}
\usepackage[savepos]{zref}
\usepackage{multicol}
\usepackage{sectsty}
\usepackage{xcolor}
\usepackage[framemethod=tikz]{mdframed}
\usepackage{hyperref}

\usepackage[paper=a4paper]{geometry}
\geometry{headheight=2.6cm,headsep=3mm,footskip=13mm}
\geometry{top=2cm,bottom=2cm,left=2cm,right=2cm}


\setCJKmainfont[BoldFont={SimHei}]{SimSun}
\newfontfamily{\monotype}{Consolas}
%\newcommand{\monotype}{\tt}

\pagestyle{fancy}

\fancyhead[L]{MH2500 TEST2 CHEET SHEET}
\fancyhead[R]{\sf\rightmark}

\setlength{\parindent}{0em}

% settings for listings
\lstset {
  basicstyle = \small\monotype,
  language = C++,
  tabsize = 2,
  breaklines = true,
  breakindent = 1.1em,
  numbers=right,
  stringstyle=\monotype,
  numberstyle=\footnotesize\ttfamily,
  firstnumber=last,
  basewidth={0.5em, 0.4em},
  frame=single
}

% an amazing script
% converts an line-number to arbitrary string
\let\othelstnumber=\thelstnumber
\def\createlinenumber#1#2{
    \edef\thelstnumber{%
        \unexpanded{%
            \ifnum#1=\value{lstnumber}\relax
             \tt #2%
            \else}%
        \expandafter\unexpanded\expandafter{\thelstnumber\othelstnumber\fi}%
    }
    \ifx\othelstnumber=\relax\else
      \let\othelstnumber\relax
    \fi
}

\usepackage{enumitem}

\setlist[itemize]{itemsep=0pt}